% Created 2018-01-25 Thu 12:55
% Intended LaTeX compiler: pdflatex
\documentclass[11pt]{article}
\usepackage[utf8]{inputenc}
\usepackage{lmodern}
\usepackage[T1]{fontenc}
\usepackage{fixltx2e}
\usepackage{graphicx}
\usepackage{longtable}
\usepackage{float}
\usepackage{wrapfig}
\usepackage{rotating}
\usepackage[normalem]{ulem}
\usepackage{amsmath}
\usepackage{textcomp}
\usepackage{marvosym}
\usepackage{wasysym}
\usepackage{amssymb}
\usepackage{amsmath}
\usepackage[version=3]{mhchem}
\usepackage[numbers,super,sort&compress]{natbib}
\usepackage{natmove}
\usepackage{url}
\usepackage{minted}
\usepackage{underscore}
\usepackage[linktocpage,pdfstartview=FitH,colorlinks,
linkcolor=blue,anchorcolor=blue,
citecolor=blue,filecolor=blue,menucolor=blue,urlcolor=blue]{hyperref}
\usepackage{attachfile}
\author{Yuanning Li, Shamindra Shrotriya}
\date{\today}
\title{Data file details}
\begin{document}

\tableofcontents


\section{Category}
\label{sec:org93645e8}

\begin{itemize}
\item In the \textbf{category task}, images from different categories are shown to the subjects
\item The neural activities from different electrodes are recorded and saved as csv files.
\item Each file contains the \textbf{2-dimensional array} of all the data recorded from \textbf{one} single electrodes.
\item In each session of the category dataset, there are 480 trials from 6 different conditions (some cases 7 conditions, see ‘Localizer\_category\_key.xlsx’ for details).
\item Each condition is a type category of images, e.g. bodies, faces, words, houses, scrambles, etc.
\end{itemize}

\subsection{Category Data - Naming Conventions}
\label{sec:orgd24bb0b}

\begin{itemize}
\item The \textbf{broadband data} files are named as:
\begin{itemize}
\item \texttt{\$\{SUBJECT\_ID\}\_Category\_Localizer\_Session\$\{SESSION\_INDEX\}\_Broadband\_Ch\$\{CHANNEL\_INDEX\}.dat}
\end{itemize}

\item The \textbf{ERP data} files are named as:
\begin{itemize}
\item \texttt{Session\$\{SESSION\_INDEX\}\_\$\{TASK\}\_data\_\$\{FILTER\}\_all\_channels\_ch\$\{CHANNEL\_INDEX\}.dat}
\end{itemize}
\end{itemize}


\subsection{Category Data - Key Naming Field Definitions}
\label{sec:org477ffbc}

The key fields in the naming convention are defined as follows:
\begin{itemize}
\item \texttt{SESSION\_INDEX} is the \textbf{session number} (1 or 2 in most cases)
\begin{itemize}
\item This is sometimes omitted if there is only one session
\end{itemize}
\item \texttt{TASK} is the task name, i.e. \textbf{category}
\item \texttt{FILTER} is the way the data is filtered, i.e. \textbf{bandpass} or \textbf{raw}
\item \texttt{CHANNEL\_INDEX} is the \textbf{electrode index} \(1,2,3, \ldots\)
\end{itemize}

\subsection{Category Data - Contents}
\label{sec:org4fa8e66}

\begin{itemize}
\item For the category task, each data file contains the \(N \times T\) data matrix from channel \(X\) (\(X = 1, 2, \ldots\))
\item Each column is a feature (time point) and each row is a trial
\item ERP:
\begin{itemize}
\item Typically \(N = 480\) trials and \(T = 1500\) time points
\item The data is sampled at \textbf{1000hz}
\end{itemize}
\item Broadband:
\begin{itemize}
\item Typically \(N = 480\) trials and \(T = 150\) time points
\item The data is sampled at \textbf{100hz}
\end{itemize}
\item All the trials are aligned to the onset time of the image stimulus
\item The 1500 time points start from 500 ms before stimulus onset and ends at 999 ms after stimulus onset.
\item The 150 time points start from 500 ms before stimulus onset and ends at 990 ms after stimulus onset.
\end{itemize}

The Category\_label.dat file contains the 480-by-1 label vector for each trial condition, the value of the i-th component in the vector represents the image stimulus type for the i-th trial, 1: bodies, 2: faces, 3: words(or shoes), 4: hammers, 5: houses, 6: non-objects. (see ‘Localizer\_category\_key.xlsx’ for details).

In addition to the single trial data, the continuous timecourse data for the category task is also included. You’ll find the file Category\_fulltimecourse\_signal.dat for each session. This is the raw time sequence for the entire session. The file is a N by L matrix where N is the number of electrodes and L is the total time length of the entire session (typically around 1e6 time points sampled at 1000 Hz). In the file Category\_timestamps.dat, the timestamps for the stimulus onset time of each trial are saved in a N-by-1 vector, where N is the total number of trials. For example, if the i-th elements in the vector is x, then the stimulus onset of the i-th trial happens at the x-th timepoint in Category\_fulltimecourse\_signal.dat. The condition for each trial is the same as Category\_label.dat.

\section{Individuation}
\label{sec:orgb8c8cd0}

\begin{itemize}
\item In the experiment the patient was passively viewing a series of faces and the task was to respond whether the face being presented was a male or female face.  - There were:
\begin{itemize}
\item 14 different face identities in total,
\item each face identity had 5 different emotions
\item with 3 different gazes.
\item Each individual picture was presented \textbf{4 times}
\end{itemize}
\item This yields 840 trials in total in one session i.e. \(840 = 14 \times 5 \times 3 \times 4\).
\end{itemize}

\subsection{Individuation - Naming Convention}
\label{sec:orgaec1d3e}

The files are named as:

\texttt{Session\$\{SESSION\_INDEX\}\_\$\{TASK\}\_data\_\$\{FILTER\}\_all\_channels\_ch\$\{CHANNEL\_INDEX\}.dat}

\subsection{Individuation Data - Key Naming Field Definitions}
\label{sec:org04bda2f}

\begin{itemize}
\item \texttt{SESSION\_INDEX} is the session number (1 or 2 in most cases), sometimes this is omitted if there is only one session
\item \texttt{TASK} is the task name, i.e. \textbf{individuation}
\item \texttt{FILTER} is the way the data is filtered, i.e. \textbf{bandpass} or \textbf{raw}
\item \texttt{CHANNEL\_INDEX} is the index of the electrode, \(1, 2, 3, \ldots\)
\end{itemize}

\subsection{Folder Structure}
\label{sec:orgf55c1e9}

\begin{itemize}
\item The files of each session are sorted into the corresponding folder. 
\item For each session, the labels for the trials are written into the following files:
\item \texttt{\$\{SESSIONID\}\_\$\{CONDITION\}\_label.csv}
\begin{itemize}
\item \texttt{SESSIONID in \{Session1, Session2\}}
\item \texttt{CONDITION in \{Face, Gender, Expression, Gaze\}}
\begin{itemize}
\item \texttt{Face} - the index of the face identity, value range 1 - 14
\item \texttt{Gender} - the index of the face gender, 1 - male, 2 - female
\item \texttt{Expression} - the index of the face expression, 1 - angry, 2 - afraid, 3 - happy, 4 - neutral, 5 - sad
\item \texttt{Gaze} - the index of the eye gaze, 1 - front, 2 - left, 3 - right
\end{itemize}
\end{itemize}
\end{itemize}
\end{document}